\chapter*{Lesson 18}

\subsection{Public Key Infrastructure}

The problem now is that Alice has a public key, but she wants "the blue check"
over it, so Bob is sure that public key comes only from the true Alice.\\

To obtain the blue check, Alice needs an universally-trusted third party, called
\textit{ Certification Authority}, which will provide a special
\textit{signature} to Alice for proving her identity to Bob.\\
The schema works as follows:

\begin{figure}[h!]
   \centering
   \sdinit{}
   \begin{tikzpicture}
      \sdbegin{}
      \newinst{A}{ Certification Authority$pk_{CA}, sk_{CA}$}
      \newinst[4]{B}{Alice}

            \postlevel
            \mess{B}{"Alice", $pk_{A}$}{A}
            \node[anchor=west] at (mess from) {\shortstack[l]{
            		$  pk_{A}, sk_{A}  $ 
                  \\
                  $  pk_{CA}  $ }};
            \postlevel
            \mess{A}{$cert_{A}$}{B}
            \node[anchor=west] at (mess to) {\shortstack[l]{where $cert_{A}$  is\\
                        $\sigma \leftarrow\$ Sign (sk_{CA}, "Alice"\|pk_{A})$  } };
      
            
      \sdend{}
      \sdend{}
   \end{tikzpicture}
\end{figure}

The CA message $pk'$ is also called as $cert_{A}$, the signature of the CA for
Alice.\\

Now, when Bob wants to check the validity of the Alice's public key:

\begin{figure}[h!]
   \centering
   \sdinit{}
   \begin{tikzpicture}
      \sdbegin{}
      \newinst{A}{$ CA $}
      \newinst[2]{B}{ Alice }
      \newinst[4]{C}{Bob}

            \postlevel
            \mess{B}{"Alice", $pk_{A}$}{A}
            \node[anchor=west] at (mess from) {\shortstack[l]{
            		$  pk_{A}, sk_{A}  $ 
                  \\
                  $  pk_{CA}  $ }};
            \postlevel
            \mess{A}{$cert_{A}$}{B}
            \node[anchor=west] at (mess to) {\shortstack[l]{where $cert_{A}$  is\\
                        $\sigma \leftarrow\$ Sign (sk_{CA}, "Alice"\|pk_{A})$  } };
    
                  \postlevel
                  \mess{B}{}{C}
                  \node[anchor=west] at (mess to) {\shortstack[l]{
                  		$  pk_{BOB}, pk_{CA} $ 
                        \\
                $  Vrfy(pk_{CA}, "Alice"\|pk_{A}, certa_{A})  $ }};
            

      \sdend{}
      \sdend{}
   \end{tikzpicture}
\end{figure}

How can Bob recognize a valid certificate from an expired/invalid one?\\

The infrastructure provides some servers which contain the lists of the
currently valid certificates.\\

\begin{theorem}
    Signatures are in \textit{ \textbf{Minicrypt} } .
\end{theorem}
This is a counterintuitive result, not proven during the lesson, but very
interesting because it implies that we can create valid signatures only with hash
functions, without considering at all public key encryption.\\


In the next episodes:
\begin{itemize}
    \item Digital Signatures from TDP*
    \item Digital Signatures from ID Scheme*
    \item Digital Signatures from CDH
\end{itemize}

Where * appears, something called \textit{Random Oracle Model} is used in the
proof. Briefly, this model assumes the existance of an ideal hash function which
behaves like a truly random function (outputs a random y as long as x was never
queried, otherwise gives back the already taken y).
