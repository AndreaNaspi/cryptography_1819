\chapter*{Lesson 19}

\subsection{Waters Signature}
%https://eprint.iacr.org/2011/703.pdf
This is a standard model for PKI which \textbf{does not need a Random Oracle}. (or at least I think so)

Let's define a \textbf{Bilinear Group} as $(\G, \G_t, q, g, \hat{e})\leftarrow BilGen(1^{\lambda})$ where:
\begin{itemize}
    \item $\G, \G_t$ are prime order groups (order $q$).
    \item $g$ is a random generator of $\G$.
    \item $(\G, \cdot)$ is a multiplicative group and $\G_t$ is called \underline{target} group.
    \item $\hat{e}$ is a (bilinear) MAP: $\G \times \G \rightarrow \G_t$ efficiently computable. Defined as follows:
    
    \centering{$\rightarrow$ Take a generator $g$ for $\G$ and an element $h$ in $\G$.\\
    $\forall g,h \in \G, a, b \in \Z_q \hat{e}(g^a,h^b)=\hat{e}(g,h)^{ab}=\hat{e}(g^{ab},h)$ \\and $\hat{e}(g,g)\neq 1$ (Non degenerative)}
\end{itemize}

Venturi said something here related to Weil pairing over an elliptic curve. I found \href{https://www.math.auckland.ac.nz/~sgal018/crypto-book/ch26.pdf}{this}. Interesting but not useful.

\noindent\textbf{Assumption:} CDH is HARD in $\G$.\\
\textbf{Observation:} DDH is EASY in $\G$!
\begin{proof}
    $(g,g^a,g^b,g^c)\approx_c (g,g^a,g^b,g^{ab}) \implies \hat{e}(g^a,g^b)=_?\hat{e}(g,g^c)$ now just move the exponents and I can transform the first element in $\hat{e}(g,g^{ab})$ and check if it is equal to $\hat{e}(g,g^c)$.
\end{proof}

Now $KGen(1^{\lambda})$ will:
\begin{itemize}
    \item Generate some params $ =(\G,\G_t,g,q,\hat{e})\leftarrow BilGen(1^{\lambda})$ 
    \item Pick $a \leftarrow\$ \Z_q$ then $g_1=g^a$
    \item Pick $g_2=g^b$ and $g_2,u_0,u_1,...,u_k \leftarrow\$ \G$.
    \item Then output: 
    \begin{itemize}
        \item $P_k=(params, g_1,g_2,u_0,...,u_k)$
        \item $S_k=g_2^a=g^{ab}$
    \end{itemize}
\end{itemize}

Sign($S_k$, m):
\begin{itemize}
    \item Divide the message $m$ of length $k$ in bits and not it as follows: $m=(m[1],...,m[k])$
    \item Now define $\alpha(m)=u_0\prod_{i=1}^k u_i^{m[i]}$
    \item Pick $r\leftarrow \Z_q$ and output the signature $\sigma =(\underbrace{g_2^a}_{S_k} \cdot \alpha(m)^r,g^r)=(\sigma_1,\sigma_2)$
\end{itemize}

Vrf($P_k$, m, ($\sigma_1,\sigma_2$))
\begin{itemize}
    \item Check $\hat{e}(g,\sigma_1)=\hat{e}(\sigma_2,\alpha(m))=\hat{e}(g_1,g_2)$
\end{itemize}

\textbf{Correctness:} \textit{"Just move the exponents"}\\ 

$\hat{e}(g,\sigma_1)=\overbrace{\hat{e}(g,g_2^a\cdot \alpha(m)^a)}^{\text{I can separate the second part for bilinearity}}=\hat{e}(g,g_2^a)\cdot \hat{e}(g,\alpha(m)^r)=\hat{e}(\underbrace{g^a}_{g_1},g_2)\cdot \hat{e}(\underbrace{g^r}_{\sigma_2},\alpha(m))=\hat{e}(g_1,g_2)\cdot\hat{e}(\sigma_2,\alpha(m))$\\
We can say that we are moving the exponents from the "private domain" to the "public domain".