\section{Textbook RSA}
This is an \underline{insecure} toy example of the more complex \textit{RSA} (Rivest Shamir Adleman) algorithm.
The key generation algorithm: $\KGen=\Gen RSA(1^{\lambda})$ outputs $P_k=(n,e)$ and $S_k=d$, then we have

\begin{gather*}
    Enc(P_k,m)=m^e\Mod{n}\\
    Dec(S_k,c)=c^d\Mod{n}
\end{gather*}

Since the output of Enc is deterministic this is \textbf{not CPA secure}! However it can be used with HARD-CORE Predicate.\\
Preprocess the message to add randomness:
$$\hat{m}=r||m \text{ where }r\leftarrow\mathdollar\{0,1\}^l$$
now Enc is not deterministic.\\
\textbf{Facts:}
\begin{enumerate}
    \item $l \in super(log(\lambda))$ otherwise it is possible to bruteforce in PPT.
    \item If $m\in\{0,1\}$ then I can prove it CPA secure under RSA (just use standard TDP)
    \item If $m$ is "in the middle" ($\{0,1\} \leq m \leq \{0,1\}^l$) RSA is believed to be secure and is \underline{standardized} (PKCS\#1,5)
    \item Still not CCA secure!
\end{enumerate}

\subsection{Trapdoor Permutation from Factoring}
Let's look at $f(x)=x^2\Mod{n}$ where $f: \Z_n^* \to \QR[n] (\subset \Z_n^*)$, this is not a permutation in general.\\
Now let's consider the Chinese Reminder Theorem (CRT) representation:

    \begin{gather*}
        x=(x_p,x_q) \rightarrow x_p\equiv x\Mod{p} , x_q\equiv x\Mod{q}\\
        f(x)=x^2\Mod{p}; x \from \$ \Z_p^*
    \end{gather*}

Since $Z_p^*$ is cyclic I can always write:

\begin{gather*}
    Z_p^*=\{g^0,g^1,g^2,\ldots,g^{\frac{p-1}{2}-1},g^{\frac{(p-1)}{2}},\ldots,g^{p-2} \}\\
    \QR[p]=\{g^0,g^2,g^4,\ldots,\overbrace{g^{p-3}}^{g^{\frac{pi1}{2}-1} in Z_p^*},\underbrace{g^0}_{g^{\frac{p-1}{2}}in Z_p^*},\ldots\}\\
    |\QR[p]|=\frac{p-1}{2}
\end{gather*}
Moreover since $(g^{\frac{p-1}{2}})^2 \equiv 1 \Mod{p} $ then $g^{\frac{p-1}{2}} \equiv -1 \Mod{p}$.\\
Now assume $p\equiv 3 \Mod{4}$ ([*]$p=4t+3$ CRT), then squaring $Mod{p}$ is a permutation because, given \underline{$y=x^2 \Mod{p}$} if I compute:
\begin{gather*}
    (y^{t+1})^2=\underbrace{y^{2t+2}}_{\text{[*] }2t+2=\frac{p-3}{2}+2=\frac{p+1}{2}=\frac{p-1}{2}+1}=(x^2)^{\frac{p-1}{2}+1}=1x^2=x^2\\
    \implies x=\pm y^{t+1}
\end{gather*}

But only 1 among the above $\pm y^{t+1}$ is a square, this is because $\frac{p-1}{2}$ is odd.

\begin{lemma}
    $\forall z, z\in \QR[p]$ IFF $-z\notin \QR[p]$
\end{lemma}