\chapter*{Lesson 10}
\section{Domain extension for PRFs/MACs}

\textit{Almost universal approach} : I have a family $\F(\H)$ with $\H$ AU and
with PRF $f \in \F$.\\

\textit{Computational AU} : we want to build a family $\H$ using some other
PRFs. We expect to have:
\begin{itemize}
    \item $ \P [ h_{s}(m)=h_{s}(m'), s\leftarrow\$\{0,1\}^{\lambda} , (m,
        m')\leftarrow\$A(1^{\lambda}) ] \in \negl{\lambda} $;
    \item We need two PRFs. One is $F_{k}$, and the other is $F_{s}j$\\
=========================================================================\\
something related to $f_{s}(1,.)$ and $f_{s}(0, . )$, but I didn't get it. \\
=========================================================================\\
\end{itemize}.

\subsection{XOR mode}
Assume that we have this function
\[
    h_{s}(m)=F_{s}(m_{1}||1) \xor ... \xor F_{s}(m_{t}||t)
\]
so that the input to the PRF $F_{s}(.)$ is $n + log_{2} t$ bytes long.
\begin{lemma}
    Above $\H$ is computational AU if $\F$ is a PRF.
\end{lemma}
\begin{exercise}
    Prove this !
\end{exercise}
Possible proof:\\

we have to show that
\[
    \P [ h_{s}(m)=h_{s}(m') ] \in \negl{\lambda}   
\]
with $m\not=m'$.\\
This means that 
    \begin{gather*}
        \P [F_{s}(m_{1}||1) \xor ... \xor F_{s}(m_{t}||t) = F_{s}(m'_{1}||1) \xor ...
        \xor F_{s}(m'_{t}||t)]  =\\
        = \P [ F_{s}(m_{i}\|i) \xor  F_{s}(m'_{i}\|i)= \alpha  = \bigxor_{j=1, j\not=i}^{t} 
        F_{s}(m_{j}\|j) \xor  F_{s}(m'_{j}\|j)]
    \end{gather*}
    for each $i \in [1,t]$.\\
    But $\alpha$ is one unique random number chosen over $2^{n}$ possible
    candidates, so the collision probability is negligible.\\
\subsection{CBC MAC}
This is part of the standard, used in TLS. It's used with a PRF $F_{s}$, setting
the starting vector $IV=0^{n}=c_{0}$ and running this PRF as part of CBC. The
output of the CBC process is just the last block:
\[
    h_{s}(m_{1}, ..., m_{t})=F_{s} ( m_{t} \xor F_{s}(m_{t-1} \xor ... \xor 
    F_{s}(m_{2} \xor F_{s}(m_{1} \xor IV))  ))
\]
\begin{lemma}
    CBC MAC defines completely an AU family.\\
    (not proven)
\end{lemma}

We can use this function to create an \textbf{encrypted CBC}, or \textbf{E-CBC}  :
\[
    E-CBC_{K, S}(m)=F_{k}(h_{s}^{CBC}(m))
\]
.
\begin{theorem}
    Actually if $\F$ is a PRF, CBC-MAC is already a MAC with domain $nt$ for
    arbitrary but fixed $t \in \mathbb{N} $.\\
    (not proven)
\end{theorem}

\subsection{XOR MAC}
Instead of $\F(\H)$ now the $Tag()$ function outputs $\phi = (\eta , F_{k}(\eta)
\xor h_{s}(m))$ where $\eta \leftarrow\$ \{0,1\}^{n}$ is random and it's called
\textit{nonce} .\\
When I want to authenticate, I should send the
\[
    (m,( \eta, F_{k}(\eta)\xor h_{s}(m)))
\]
couple.\\
When I want to verify a message and I get the couple $(m, (\eta, v))$, I just
check that $v=F_{k}(\eta) \xor h_{s}(m)$. It should be hard to find a value
called $a$ such that, given $m \not= m'$, 
\[
    h_{s}(m) \xor a= h_{s}(m')
\]
In fact, since an adversary who wants to break this scheme has to send a valid
couple $(m^{*}, \phi^{*})$ after some queries, he could:
\begin{itemize}
    \item ask for message $m$ and store the tag $( \eta, F_{k}(\eta)\xor h_{s}(m))$
    \item try to find $a=h_{s}(m) \xor h_{s}(m')$ and modify the previous stored
        tag adding $v \xor a$, 
\end{itemize}
so now he could send the authenticated message
\[
    (m', (\eta, F_{k}(\eta) \xor h_{s}(m'))) 
\]
which is a valid message.\\

==============================================================\\
WHAT IS IT ABOUT?
\begin{lemma}
    XOR mode gives computational AXU (Almost Xor Universal).\\
    (not proven)
\end{lemma}
WAHT DOES IT MEAN AXU?? Has something to do with the first 2 definitions given
in the start of this subsection ?
==============================================================\\

\begin{theorem}
    If $\F$ is a PRF and $\H$ is computational AXU, then XOR-MAC is a MAC.\\
    (not proven)
\end{theorem}

==============================================================\\
NOT CLEAR WHAT TO DO 
\begin{exercise}
    Now with variable input lenght:
    \begin{itemize}
        \item AXU based XOR mode 
        \item $\F(\H)$ is insecure with polynomial construction
            $h_{s}(m)=q_{m}(s)$, but can be fixed.
        \item CBC-MAC not secure. (exercise)
        \item E-CBC is secure.
    \end{itemize}
==============================================================\\
    
\end{exercise}
\newpage
\section{Chosen Ciphertext Attack security}
In this kind of attack, the adversary has a decryption capability added to
the previous encryption capability. A scheme which is CCA-secure is also
\textbf{non-malleable}, since the attacker cannot modify the obtained
ciphertexts to obtain new valid ciphertexts.\\
\begin{figure}[h!]
   \centering
   \sdinit{}
   \begin{tikzpicture}[scale=0.7]
      % Define symbols and names for the parties
      \sdbegin{}
      \newinst{A}{$ \A $}
      \newinst[7]{B}{$ C^{cca}_{k \leftarrow\$ \K} $} % Increase "5" to widen
      
      \postlevel
      \mess{A}{$m$}{B}
      \node[anchor=west] at (mess to) {  };
      \postlevel
      \mess{B}{$c$}{A}
      \node[anchor=west] at (mess from) {$c\leftarrow\$Enc(k, m)$ or $c=Enc(k,
          m, r)$};
    \draw [->] (1.2,-3.3) to[out=240,in=110] (1.2,-1.7);
      
    \postlevel
      \mess{A}{$c'$}{B}
      \node[anchor=west] at (mess to) {  };
      \postlevel
      \mess{B}{$m'$}{A}
      \node[anchor=west] at (mess from) {$m'=Dec(k, c')$ };
    \draw [->] (1.2,-5.6) to[out=240,in=110] (1.2,-4);
      
   \postlevel 
    \mess{A}{$m_{0}^{*}, m_{1}^{*}$}{B}
      \node[anchor=east] at (mess from) {  };
      \postlevel
      \mess{B}{$c^{*}$}{A}
      \node[anchor=west] at (mess from) {\shortstack[l]{
        $r^{*} \leftarrow\$ \{0,1\}^{*} 
        $    
          \\
          $c^{*}=Enc(k, m_{b}^{*}, r^{*})$
          }};
      
      % Message from Bob to Alice, with computations by both sides
    \postlevel
      \mess{A}{$m$}{B}
      \node[anchor=west] at (mess to) {  };
      \postlevel
      \mess{B}{$c$}{A}
      \node[anchor=west] at (mess to) {  };
    \draw [->] (1.2,-10.5) to[out=240,in=110] (1.2,-9);


    \postlevel
      \mess{A}{$c'$}{B}
      \node[anchor=west] at (mess to) {  };
      \postlevel
      \mess{B}{$m'$}{A}
      \node[anchor=west] at (mess from) { };
    \draw [->] (1.2,-13) to[out=240,in=110] (1.2,-11.5);



      \postlevel
      \mess{A}{$b' \in \{0,1\}$}{B}
      \node[anchor=west] at (mess to) {  };
      \sdend{}
   \end{tikzpicture}
   \caption{ $Game_{\Pi, \A}^{cca}(\lambda, b)$}
\end{figure}

\begin{exercise}
    Show that $(r, F_{k}(r) \xor m)$ which is CPA-secure, is not CCA-secure.
\end{exercise}
\begin{proof}
    \begin{figure}[h!]
       \centering
       \sdinit{}
       \begin{tikzpicture}[scale=0.8]
          % Define symbols and names for the parties
          \sdbegin{}
          \newinst{A}{$ \A $}
          \newinst[5]{B}{$ C $} % Increase "5" to widen
          
          \mess{A}{$0^{n}, 1^{n}$}{B}
          \node[anchor=west] at (mess to) {  };
    
          \postlevel
          \mess{B}{$c_{\alpha}=(r_{\alpha},s_{\alpha} )$}{A}
          \node[anchor=west] at (mess from) {$c_{\alpha}=(r_{\alpha},
              F_{k}(r_{\alpha} \xor m_{b}))$};
          
          % Message from Bob to Alice, with computations by both sides
          \postlevel
          \mess{A}{$c=(r_{\alpha}, s_{\alpha} \xor 1000...000)$}{B}
          \node[anchor=west] at (mess to) {$Dec(k, c)$  };
          
          \sdend{}
       \end{tikzpicture}
    \end{figure}
    with 
    
    \begin{gather*}
        Dec(k, c) =  F_{k}(r_{\alpha}) \xor s_{\alpha} \xor
        1000...000 =\\
        F_{k}(r_{\alpha}) \xor F_{k}(r_{\alpha}) \xor m_{b} \xor
        1000...000 =\\
        m_{b} \xor 1000...000 
    \end{gather*}
   At this point we have that the output is  
   \begin{itemize}
       \item $1000...000$ if $m_{b}$ was $000...000$
       \item $0111...111$ if $m_{b}$ was $0111...111$
   \end{itemize}
   
\end{proof}

\section{Authenticated encryption}
\textbf{Idea}: what if we combine the target of authenticity with the
target of encryption?\\
The first property is satisfied when the receiver is able to understand if the
received message was sent exactly by the trusted sender; the last property is
satisfied when no information of the sent message is contained in the ciphertext,
thus only the chosen receiver can fully read the original sent message.\\
If we build up a schema with these 2 properties, we can obtain a new schema
which should be \textit{cpa-secure} (to enforce the encryption/privacy property)
and essentially secure against forgeries of chosen message attacks (to enforce
the authentication property).\\
In particular, a successful forgery can be obtained winning the following game:

\begin{figure}[h!]
   \centering
   \sdinit{}
   \begin{tikzpicture}
      % Define symbols and names for the parties
      \sdbegin{}
      \newinst{A}{$ \A $}
      \newinst[5]{B}{$ C_{k \leftarrow\$ \K} $} % Increase "5" to widen
      
      \mess{A}{$m$}{B}
      \node[anchor=west] at (mess to) {  };

      \postlevel
      \mess{B}{$c=Enc(k, m)$}{A}
      \node[anchor=west] at (mess from) {\shortstack[l]{
      		$    $ 
            \\
            $    $ }};
      
      \postlevel
      \mess{A}{$c^{*}$}{B}
      \node[anchor=west] at (mess to) {\shortstack[l]{
                  Output 1 IFF $Dec(k, c^{*})\not=\perp$ and
            \\
        $c^{*} \not\in \{c\}$ }};

    \draw [->] (1.2,-2.7) to[out=240,in=110] (1.2,-1.0);
   \end{tikzpicture}
   \caption{$Game _{\Pi, \A}^{auth}(\lambda)$}
\end{figure}
meaning that the $c^{*}$ challenge ciphertext should be a \textbf{fresh} and a
valid one ciphertext, considering
that the $Dec(.)$ function is defined as
\[
    Dec:\K*\C \to M \cup \{\perp\} 
\]
where $\perp$ represents invalid/meaningless messages.\\

In the end, we want that our \textit{authenticated encryption schema} is
cpa-secure and has this last property, called \textbf{strong unforgeability} or
\textbf{auth} . 

\begin{theorem}
    Let $\Pi$ be an SKE (shared key encryption). If $\Pi$ has \textbf{
    \textit{CPA + auth} } security, then it also has \textbf{CCA} security.
\end{theorem}

\begin{exercise}
    Prove it!\\
    \textbf{Hint}: consider the experiment where $Dec(k, c)$:
    \begin{itemize}
        \item if $c$ not fresh ( i.e. output of previous encryption query $m$,
            output $m$)
        \item else output $\perp$
    \end{itemize}
    .
    ====================================================\\
    TO DO AND TO PROVE\\
    Approach: reduce cca to cpa; given $D^{cca}$, we can build $D^{cpa}$.
    $D^{cca}$ will ask decryption queries, but $D^{cpa}$ can answer just with
    these two properties shown above, so it can reply just if he asked these
    $(c, m)$ before to its challenger $\C$.
    ====================================================\\
\end{exercise}

\subsection{Three approaches to authenticated encryption}
Let $\Pi_{1}$ be a \textbf{cpa-secure} SKE and $\Pi_{2}$ be a \textbf{auth} MAC
schema.\\
We have 3 ways to combine these 2 schema to obtain a new one:
\begin{enumerate}
    \item \textbf{ \textit{Encrypt-and-MAC} } :
        \begin{gather*}
            c \leftarrow\$ Enc(k_{1}, m)\\
            \phi = Tag(k_{2}, m)\\
            c^{*}=(c, \phi)
        \end{gather*}
        
    \item  \textbf{ \textit{MAC-then-encrypt} } :
        \begin{gather*}
            \phi = Tag(k_{2}, m)\\
            c \leftarrow\$ Enc(k_{1}, \phi \| m)\\
            c^{*}=c
        \end{gather*}
        
    \item  \textbf{ \textit{Encrypt-then-MAC} } :
        \begin{gather*}
            c \leftarrow\$ Enc(k_{1}, m)\\
            \phi = Tag(k_{2}, c) \\
            c^{*}=(c, \phi)
        \end{gather*}
        
\end{enumerate}
In a paper is clearly stated that the first solution doesn't work taking random
combination of CPA-secure and MAC schemes, because there are couple which,
mixed, are not CCA-secure. Furthermore the second solution doesn't work for the
same above reason, even if it's now part of the standard.\\

Instead, the third solution works always for a CPA-secure scheme $\Pi_{1}$ and
strong unforgeability scheme $\Pi_{2}$ which, combined, they generate the final scheme
$\Pi$.
\begin{theorem}
    If $\Pi$ is made combining $\Pi_{1}$ cpa-secure and $\Pi_{2}$
    \textit{auth}-secure 
    as stated in the third point,
    then $\Pi$ is cpa-secure and \textit{auth-secure} . 
\end{theorem}
\begin{proof}
    By reduction, we negate that $\Pi$ has both the properties and we find
    the contraddiction.\\
    Suppose now that $\Pi$ hasn't the cpa-security property.



\begin{figure}[h!]
   \centering
   \sdinit{}
   \begin{tikzpicture}[scale=0.5]
      % Define symbols and names for the parties
      \sdbegin{}
      \newinst{A}{$ \A' $}
      \newinst[5]{B}{$ \A $} % Increase "5" to widen
      \newinst[4]{C}{$ \C_{k \leftarrow\$ \{0,1\}^{\lambda} 
      } $}

      \postlevel
      \mess{A}{$m$}{B}
      \node[anchor=west] at (mess to) {  };
      \postlevel
      \mess{B}{$m$}{C}
      \node[anchor=west] at (mess to) {  };

      \postlevel
      \mess{C}{$c'$}{B}
      \node[anchor=west] at (mess from) {\shortstack[l] {
                  $c'\leftarrow\$ Enc(k, m)$
          } };
      \postlevel
      \mess{B}{$(c',\phi=Tag(k_{2}, c')) $}{A}
      \node[anchor=west] at (mess to) {  };
    \draw [->] (1.2,-5.7) to[out=240,in=110] (1.2,-1.6);
    \postlevel
    \mess{A}{$m_{0}^{*},m_{1}^{*} $}{B}
      \node[anchor=west] at (mess to) {  };

      \postlevel
    \mess{B}{$m_{0}^{*}, m_{1}^{*}$}{C}
      \node[anchor=west] at (mess to) {  };
    
    \postlevel
    \mess{C}{$c^{*}$}{B}
      \node[anchor=west] at (mess to) {  };
    
      \postlevel
      \mess{B}{$c^{*}, \phi^{*}$}{A}
      \node[anchor=west] at (mess from) {\shortstack[l]{
      		$    $ 
            \\
            $    $ }};
    \postlevel
    \mess{A}{other messages}{C};
    \postlevel
    \mess{C}{other responses}{A};

    \postlevel
    \mess{A}{$b'$}{B}
      \node[anchor=west] at (mess to) {  };
      
    \postlevel
    \mess{B}{$b'$}{C}
      \node[anchor=west] at (mess to) {  };
      
%      \postlevel
%      \mess{A}{$b' \in {0,1}$}{B}
%      \node[anchor=west] at (mess to) {  };
      \sdend{}
      \sdend{}
   \end{tikzpicture}
  % \caption{}
  % \label{fig:}
\end{figure}

Here we are supposing that $\A$ can break the cpa-security of a generic $\Pi_{1}$ 
scheme used by $\C$, while $\A'$ can break the cpa-security of a generic scheme
$\Pi$. $\C$ can generate the $Tag(k_{2},.)$ function, randomly chosing
$k_{2}$((((
and simulating $\Pi$.
\\=================================================================================\\
REVIEW\\
Professor says that we have to show that $Game^{cpa}(\lambda, 0) \approx_{c}
Game^{cpa}(\lambda, 1)$ , but why??? Isn't this proof enough?
\\=================================================================================\\
Proved for the cpa-security property, now we have to prove, in a similar way,
that the auth property must be holded by $\Pi$ if $\Pi_{2}$ is an auth-secure
scheme.
\begin{exercise}
    Prove it!\\
    Similar to the cpa-security proof.
\end{exercise}





\end{proof}
