\chapter*{Lesson 5}

\section{Stretching a PRG}
\begin{theorem}
    If there exists a PRG $G:\{0,1\}^{\lambda} \to \{0,1\}^{\lambda + 1}$ ,
    then $ \forall l(\lambda) \in \poly{\lambda} $
    there exists a PRG with stretch $G:\{0,1\}^{\lambda} \to \{0,1\}^{\lambda +
    l(\lambda)} $ 
\end{theorem}

\begin{proof}
    So, consider this algorithm/construction:\\
    \begin{enumerate}
        \item Let $s_{0} \leftarrow\$ \{ 0,1\}^{\lambda}$
        \item $ \forall i \in [l] $ , let $(s_{i}, b_{i}) = G(s_{i-1})$
        \item Output $b_{1}, b_{2}, ..., b_{l}, s_{l}$ , so the output is a
            string of bit $\lambda + l(\lambda)$ long
    \end{enumerate}.\\
    Following this algorithm, we should obtain the following representation of
    $G^{l}$:
    \begin{figure}[h!]
        
\tikzstyle{int}=[draw, minimum size=2em]
\tikzstyle{dots}=[minimum size=2em]
\tikzstyle{init} = [pin edge={-to,thin,black}]

\begin{tikzpicture}[node distance=2.5cm,auto,>=latex']
    \node [int, pin={[init]above:$b_{1}$}] (a) {$G$};
    \node (b) [left of=a,node distance=1.5cm, coordinate] {a};

    \node [int, pin={[init]above:$b_{2}$}] (c) [right of=a,node distance=2cm] {$G$};
    \node [dots] (d) [right of=c,node distance=1.5cm] {$...$};

    \node [int, pin={[init]above:$b_{i+1}$}] (e) [right of=d,node distance=2cm] {$G$};
    \node [dots] (f) [right of=e,node distance=2cm] {$...$};
    \node [int, pin={[init]above:$b_{l}$}] (g) [right of=f,node distance=2cm] {$G$};

    \node [] (end) [right of=g, node distance=2cm]{};

    \path[->] (b) edge node {$s_{0}$} (a);
    \path[->] (a) edge node {$s_{1}$} (c);
    \path[->] (c) edge node {$s_{2}$} (d);
    \path[->] (d) edge node {$s_{i}$} (e);
    \path[->] (e) edge node {$s_{i+1}$} (f);

    \path[->] (f) edge node {$s_{l-1}$} (g);
    \draw[->] (g) edge node {$s_{l}$} (end) ;
\end{tikzpicture}
\caption{graphical representation of $G^{l}$}
\label{fig:gpowerl}
\end{figure}.

To show that the theorem is valid, we will use a proof by contraddiction: trying
to show that $G^{\lambda}$ is not a PRG, we will see that G should be not a PRG,
contraddicting the theorem.\\

Different sources give different interpretations of this proof. As of now , the
following one is an outline of the proof that seems to make sense, more or less.
This proof should , however, be marked as TO BE REVIEWED.\\

The step points of this proof are:
\begin{enumerate}
    \item\label{p:one} Prove that $H_{\lambda}^{i} \approx_{c} H_{\lambda}^{i+1}$ , $ \forall
        i \in [0,l]$ ;
    \item \label{p:two}Prove the \textbf{hybrid argument}: if $X \approx_{c} Y$ and $Y
        \approx_{c} Z$, then $X \approx_{c} Z$ ;
    \item \label{p:three}with the hybrid argument, prove that 
        \[
            G^{l}(U_{\lambda})= H_{\lambda}^{0} \approx_{c} 
            H_{\lambda}^{1} \approx_{c} ... \approx_{c} 
            H_{\lambda}^{l} = U_{l+\lambda}
        \] 
    \item \label{p:four} now, since $H_{\lambda}^{i}
        \approx_{c} U_{\lambda + l}$,
         it's possible to use the contraddiction
        (masked as a proof by reduction)
\end{enumerate}.\\

To prove point \ref{p:one}, define the following names:
\begin{itemize}
    \item $H_{\lambda}^{0} := G^{l}(U_{\lambda})$
    \item 
            \[  H_{\lambda}^{i}:=\begin{cases}
                        b_1 , ..., b_{i} \leftarrow\$ \{0,1\} \\
                        s_{i} \leftarrow \{0,1\}^{\lambda} \\
                        (b_{i+1}, ..., b_{l}, s_{l}) := G^{l-i}(s_{i})
                        \end{cases} 
            \] 
            

    \item  $H_{\lambda}^{l} := U_{\lambda + l}$
\end{itemize}.\\

So $H^{i}$ is just taking in input the number $x_{i}$ , executing $l-i$ times
$G$ and obtaining a sequence of bytes.\\

Now, just have a look at those figures:


    \begin{figure}[h!]
        
\tikzstyle{int}=[draw, minimum size=2em]
\tikzstyle{dots}=[minimum size=2em]
\tikzstyle{empty}=[minimum size=2em]
\tikzstyle{init} = [pin edge={-to,thin,black}]

\begin{tikzpicture}[node distance=2.5cm,auto,>=latex']
    \node [empty, pin={[init]above:$b_{1}$}] (a) {};

    \node [empty, pin={[init]above:$b_{2}$}] (c) [right of=a,node distance=1cm] {};

    \node [dots, pin={[init]above:$...$}] (r) [right of=c,node distance=1cm] {};
    \node [int, pin={[init]above:$b_{i+1}$}] (d) [right of=r,node distance=2cm] {$G$};
    \node [int, pin={[init]above:$b_{i+2}$}] (e) [right of=d,node distance=2cm] {$G$};
    \node [dots] (f) [right of=e,node distance=2cm] {$...$};
    \node [int, pin={[init]above:$b_{l}$}] (g) [right of=f,node distance=2cm] {$G$};

    \node [] (end) [right of=g, node distance=2cm]{};

    \path[->] (r) edge node {$s_{i}$} (d);
    \path[->] (d) edge node {$s_{i+1}$} (e);
    \path[->] (e) edge node {$s_{i+2}$} (f);

    \path[->] (f) edge node {$s_{l-1}$} (g);
    \draw[->] (g) edge node {$s_{l}$} (end) ;
\end{tikzpicture}
\caption{$H_{\lambda}^{i}$}
\begin{tikzpicture}
    \draw[line width=0.2 mm] (0,0) -- (12,0);
\end{tikzpicture}
      
\tikzstyle{int}=[draw, minimum size=2em]
\tikzstyle{dots}=[minimum size=2em]
\tikzstyle{empty}=[minimum size=2em]
\tikzstyle{init} = [pin edge={-to,thin,black}]

\begin{tikzpicture}[node distance=2.5cm,auto,>=latex']
    \node [empty, pin={[init]above:$b_{1}$}] (a) {};

    \node [empty, pin={[init]above:$b_{2}$}] (c) [right of=a,node distance=1cm] {};

    \node [dots, pin={[init]above:$...$}] (r) [right of=c,node distance=1cm] {};
    \node [empty, pin={[init]above:$b_{i+1}$}] (d) [right of=r,node distance=2cm] {};
    \node [int, pin={[init]above:$b_{i+2}$}] (e) [right of=d,node distance=2cm] {$G$};
    \node [dots] (f) [right of=e,node distance=2cm] {$...$};
    \node [int, pin={[init]above:$b_{l}$}] (g) [right of=f,node distance=2cm] {$G$};

    \node [] (end) [right of=g, node distance=2cm]{};

    \path[->] (d) edge node {$s_{i+1}$} (e);
    \path[->] (e) edge node {$s_{i+2}$} (f);

    \path[->] (f) edge node {$s_{l-1}$} (g);
    \draw[->] (g) edge node {$s_{l}$} (end) ;
\end{tikzpicture}
\caption{$H_{\lambda}^{i+1}$}
\end{figure}.\\

$H^{i}$ and $H^{i+1}$ differ just for the input given to the $(i+1)$-th step of the
algorithm:
\begin{itemize}
    \item in $H^{i}$, this input is pseudorandom;
    \item in $H^{i+1}$, this input comes from $U_{\lambda}$
\end{itemize}.\\

Now consider a function \footnote{\small{Why do we define $f_{i}$? Such that we know that
the first input given to $G$ in the function will be considered $s_{i+1}$.}}
\[
    f_{i}(s_{i+1},b_{i+1})=
    \left\{
        \begin{array}
            \left{b_{1}, ..., b_{i} \leftarrow \$ \{0,1\}}\\
            \forall j \in \{i+2, l\}, G(s_{j-1})=\{s_{j}, b_{j}\} \\
            \text{output :=} \{b_{1}, ..., b_{l}, s_{l}\}
        \end{array}
    \right.
\]

Given this function , it's possible to notice that:
\begin{itemize}
    \item $f_{i}(U_{\lambda + 1})$ has the same distribution of $H^{i+l}$
    \item $f_{i}(G(U_{\lambda}))$ has the same distribution of $H^{i}$
\end{itemize}.\\
Given \ref{lem:tria}, since by definition 
\[
    G(U_{\lambda}) \approx_{c} U_{\lambda + 1}
\]
then also
\[
    f_{i}(G(U_{\lambda })) \approx_{c} f_{i}(U_{\lambda + 1}) 
\]
and so $H^{i} \approx_{c} H^{i + 1}$ .\\

Now ,to prove \ref{p:two} :

\begin{align}
    X \approx_{c} Z \Rightarrow \\
    | \P [ D(X)=1 ] - \P [ D(Z)=1 ]   | \leq \nu(\lambda) \\
    | \P [ D(X)=1 ] - \P [ D(Y)=1 ] + \P [ D(Y)=1 ] -  \P [ D(Z)=1]   | \leq\\
    \leq | \P [ D(X)=1 ] - \P [ D(Y)=1 ]  | + | \P [ D(Y)=1 ] - \P [ D(Z)=1 ]| \\
    \leq \nu(\lambda) + \nu(\lambda) = \nu(\lambda) 
\end{align}

Now, to prove \ref{p:three}, it's just needed to notice that 

\begin{equation} \label{eq:hiequll}
        H_{\lambda}^{i} \approx_{c} ... \approx_{c}  H_{\lambda}^{l-1} \approx_{c} H_{\lambda}^{l} \approx_{c}  U_{l+\lambda}
    \end{equation}
    

for what's valid in point \ref{p:one} and point \ref{p:two}.\\

Now, use a contraddiction.\\
Suppose $G^{l}$ is not a a PRG $\Rightarrow$ \\

\begin{gather*}
    G^{l}(U_{\lambda}) \not\approx_{c} U_{\lambda + l} = H^{l} \not\approx_{c}
    H^{0} \Rightarrow  \\
    \exists i \in [0,l] , \exists PPT.D'  , p'(\lambda) \in \poly{\lambda} \\
    | \P [ D'(H^{i})=1 ] - \P [ D'(H^{i+l})=1 ]   | \geq \frac{1}{p'(\lambda)}
\end{gather*}.\\
This formula comes from observing that, since $H^{l} \not \approx_{c} H^{0}$,
there must be a point in the chain $H^{0} \approx_{c}  H^{1} \approx_{c} ...
\approx_{c} H^{l}$ where $H^{i} \not \approx_{c} H^{i+1}$; so
there exist $D'$ capable of distinguish them.\newpage
\begin{figure}[h!]
   \centering
   \sdinit{}
   \begin{tikzpicture}
      % Define symbols and names for the parties
      \sdbegin{}
      \newinst{A}{$ \D' $}
      \newinst[5]{B}{$ $} % Increase "5" to widen
      
      % Message from Alice to Bob, with precomputations
      \postlevel
      \mess{B}{$(b_{1}, b_{2}, ...,b_{l}, s_{l})$}{A}
      \node[anchor=west] at (mess from) {\shortstack[l]{
      		$  z \leftarrow\$H^{i}  $ \\
            $  z \leftarrow\$ H^{i+1}  $}};
      
      % Message from Bob to Alice, with computations by both sides
      \postlevel
      \mess{A}{sourcename}{B}
      \node[anchor=west] at (mess to) {  };
      
      \sdend{}
   \end{tikzpicture}
   \caption{Distinguisher for $H^{i}$ and $H^{i+1}$}
   \label{fig:disths1}
\end{figure}
If such a distinguisher exists, it can be also used to distinguish the output of
funciton $G$ from $U_{\lambda + 1}$:


\begin{figure}[h!]
   \centering
   \sdinit{}
   \begin{tikzpicture}
      % Define symbols and names for the parties
      \sdbegin{}
      \newinst{D}{$\D '$}
      \newinst[3]{A}{$ \D $}
      \newinst[3]{B}{$C^{prg} $} % Increase "5" to widen
      
      % Message from Alice to Bob, with precomputations
      \postlevel
      \mess{B}{$z=(s_{i+1}, b_{i+1})$}{A}
      \node[anchor=west] at (mess from) {\shortstack[l]{
                  $  z \leftarrow\$ G(s*) $ for $s* \leftarrow\$ \{0,1\}^{\lambda}$
            \\
            $  z \leftarrow\$ U_{\lambda +1}  $ }};
      
        \mess{A}{$f_{i}(z)=(b_{1}, ..., b_{l}, s_{l})$}{D}
      \node[anchor=west] at (mess to) {  };
      % Message from Bob to Alice, with computations by both sides
      \postlevel
      \mess{D}{$H^{i}$ or $H^{i+1}$}{A}
      \node[anchor=west] at (mess to) {  };

       \mess{A}{$G(s*)$ or $U_{\lambda+1}$}{B}
      \node[anchor=west] at (mess to) {  };
      
      \sdend{}
   \end{tikzpicture}
   \caption{\small{If $(s_{i+1}, b_{i+1})$ comes from $G(s*)$,$D'$ finds $H^{i}$,
   otherwise it finds $H^{i+1}$}}
   \label{fig:red1}
\end{figure}
So we have a contraddiction, because we cannot distinguish a PRG , by
definition.
\end{proof}
\section{Hard-core predicate}

Now , consider a tipical one-way function $f$, s.t. $f(x)=y$.
\begin{question}
    Which  bits of the input x are hard to compute given $y=f(x)$?
    Is always true that , given $f$ and $f(x)$, the first bit is hard to compute
    for every $x$?
\end{question}

\begin{example}
    Given an OWF $f$, then $f'(x)= x[0]||f(x)$ is a OWF.
\end{example}

\begin{defn}
    A polinomial time function $ h:\{0,1\}^{n} \to \{0,1\} $ is \textbf{hard
    core} for $ f:\{0,1\}^{n} \to \{0,1\}^{n} $ if 
    \begin{gather*}
        \forall.PPT. A, \exists \nu(\lambda) \in \negl{\lambda} \text{  such
        that}\\
        \P [ A(f(x))= h(x)|x \leftarrow\$ \{0,1\}^{n} ] \leq \nu(\lambda)      
    \end{gather*}
    
\end{defn}
\newpage

\begin{figure}[h!]
   \centering
   \sdinit{}
   \begin{tikzpicture}
      % Define symbols and names for the parties
      \sdbegin{}
      \newinst{A}{$ \A $}
      \newinst[5]{B}{$ C_{f,h} $} % Increase "5" to widen
      
      % Message from Alice to Bob, with precomputations
      \postlevel
      \mess{B}{z}{A}
      \node[anchor=west] at (mess from) {\shortstack[l]{
                  $  y=f(x)  $ 
            \\
        $  x \leftarrow\$\{0,1\}^{n}  $ }};
      
      % Message from Bob to Alice, with computations by both sides
      \postlevel
      \mess{A}{b}{B}
      \node[anchor=west] at (mess to) {win if $b=h(x)$  };
      
      \sdend{}
   \end{tikzpicture}
   \caption{Hard-core function, game of definition 1}
   \label{fig:def1hcf}
\end{figure}


There is also an alternative definition:
\begin{defn}
    A Polinomial Time function $h:\{0,1\}^{n} \to \{0,1\} $ is hard-core for $f$
    if
    \[
        (f(x), h(x)) \approx_{c} (f(x), b)
    \]
    where $x \leftarrow\$\{0,1\}^{n}$ and $b \leftarrow\$\{0,1\}$.
\end{defn}


\begin{figure}[h!]
   \centering
   \sdinit{}
   \begin{tikzpicture}
      % Define symbols and names for the parties
      \sdbegin{}
      \newinst{A}{$ \A $}
      \newinst[5]{B}{$ C_{f,h} $} % Increase "5" to widen
      
      % Message from Alice to Bob, with precomputations
      \postlevel
      \mess{B}{$(y,z)$}{A}
      \node[anchor=west] at (mess from) {\shortstack[l]{
                  $ y=f(x), x \leftarrow\$ \{0,1\}^{n}   $ 
            \\
    $ z \leftarrow\$ \{0,1\}   $
          \\
  $z \leftarrow\$ h(x)$}};
      
      % Message from Bob to Alice, with computations by both sides
      \postlevel
      \mess{A}{sourcename}{B}
      \node[anchor=west] at (mess to) {  };
      
      \sdend{}
   \end{tikzpicture}
   \caption{Hard-core function, game of definition 2}
   \label{fig:def2hcf}
\end{figure}

\begin{clm}
    There is no \textit{universal} hard-core function $h$.
\end{clm}
A good $h$ should be chosen for each different one-way function $f$.\\
Immagine $h$ that works for all of the OWFs.\\
What about $f'(x)=h(x)||f(x)$? If $h$ is hardcore for $f$ and $f'$, by the
definition 1 of \textbf{hardcore function} $h$ is applied on the same $x$ and
will return the same bit in $\{0,1\}$ at every interrogation.\\
TO BE REVIEWED.

\begin{theorem}[Goldreich-Levin, '99]
    Let $f$ be an OWF and consider $g(x,r)=(f(x), r)$ for $r \in \{0,1\}^{n}$.
    Then $g$ is a OWF and 
    \[ h(x,r)=<x,r>= \sum_{\text{i-th bit}}^{}x_{i}r_{i} mod2 = ...\]
    ======================================================================\\
    TO BE COMPLETED\\
    ======================================================================
    is hard core for $g$.
\end{theorem}

\begin{exercise}
    Prove that $g$ is OWF if $f$ is OWF. (by reduction)
\end{exercise}

\section{One Way Permutation}
$f:{\{0,1\}^{n}} \to {\{0,1\}^{n}} $ is an OWF and 
\[
    \forall x, |x|=|f(x)| \wedge  x \not= x' \Rightarrow f(x) \not= f(x')
\]

\begin{cor}
    If $f:\{0,1\}^{n} \to \{0,1\}^{n} $ is a OWP, then for $g(), h()$ as in the
    GL theorem,
    \[
        G(s)=(g(s), h(s))
    \]
    is a PRG.
\end{cor}

\begin{proof}
    By GL , if $f$ is an OWP, so is $g$. This means that if we want to invert
    $g$, since $g$ depends on $f$ we have to invert a OWP.\\
    Moreover $h$ is hardcore for $g$.
    Hence
    \[
        G(U_{2n}) \equiv (g(U_{2n}), h(U_{2n})) \equiv \underbrace{(f(U_{n}), U_{n},
        h(U_{2n})) \approx_{c}(f(U_{n}), U_{n}, U_{1})}_{\text{definition 1 of hard core pred.}}  \equiv U_{2n+1}
    \]

\end{proof}

We are stretching just 1 bit, but we know we can stretch more than one.
\pagebreak
