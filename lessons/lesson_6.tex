\chapter*{Lesson 6}

\section{Computationally secure encryption}

\begin{question}
    How to define the concept of \textbf{computationally secure encryption} ?    
\end{question}

Find a task/scheme that is computationally hard for an attacker to break
(supposing the attacker is $ \poly{\lambda} $, we want a scheme which requires
an amount of time near ,as much as possible, to $superpoly(\lambda)$ to be
broken).

This scheme should have these properties:
\begin{itemize}
    \item \label{prop:owk} \textbf{one wayness} w.r.t. key (given $c=Enc(k,m)$,
        it should be hard to recover k) 
    \item \label{prop:owm} \textbf{one wayness}  w.r.t. message (given
        $c=Enc(k,m)$, hard to obtain the message)
    \item \label{prop:nol} no \textbf{information leakage} about the message
\end{itemize}.

Consider the following experiment for $\Pi=(Enc, Dec)$, named 

\[
    GAME_{\Pi, \A}^{ind}(\lambda, b)
\]


\begin{figure}[h!]
   \centering
   \sdinit{}
   \begin{tikzpicture}
      % Define symbols and names for the parties
      \sdbegin{}
      \newinst{A}{$ \A $}
      \newinst[5]{B}{$ C^{ind} $} % Increase "5" to widen
      
      % Message from Alice to Bob, with precomputations
      \postlevel
      \mess{A}{$m_{0}, m \in \M$}{B}
      \node[anchor=west] at (mess to){};

      \postlevel
      \mess{B}{c}{A}
      \node[anchor=west] at (mess from) {\shortstack[l]{
      		$  k \leftarrow\$ \K  $ \\
            $ b \leftarrow\$ \{0,1\}$ \\
    $  c \leftarrow\$ Enc(k, m_{b})  $ }};
      
      % Message from Bob to Alice, with computations by both sides
      \postlevel
      \mess{A}{$b'$}{B}
      \node[anchor=west] at (mess to) {  };
      
      \sdend{}
   \end{tikzpicture}
   \label{fig:knowmess}
\end{figure}

In the image \label{fig:knowmess} $b'$ means that if $m_{b'}=m_{b}$ the
adversary wins.
\begin{definition}
    We say that $\Pi$ is computationally \textbf{one time secure }  if 
    \[
        Game_{\Pi, \A}^{ind}(\lambda, 0) \approx_{c} Game_{\Pi, \A}
        ^{ind}(\lambda,1)
    \]  
    or, alternatively $ \forall.PPT.\A \exists \nu(\lambda) \in \negl{\lambda}$
    \[
        | \P [ Game_{\Pi, \A}^{ind}(\lambda, 0)=1 ] - \P [ Game_{\Pi,
        \A}^{ind}(\lambda, 1)=1 ]   |   \leq  \nu(\lambda) 
    \]\footnote{$Game^{ind}$ refers to the indistinguishability of the messages
sent by the attacker during the game}
\end{definition}



This last definition is compliant with the three properties before exposed, in
particular:\\

if a scheme is \textbf{one time secure} $\Rightarrow$ the scheme has
each one of these 3 properties

\begin{itemize}
    \item \textbf{compliance with point 1} : suppose point 1 is not valid, and
        $k$ is not hard to discover for $ \Adv $. But then $ \Adv $ is able to perfectly
        distinguish $m$ and $m_{0}$ with $ \P [ 1 ]  $ every time, and so the
        scheme couldn't be one time secure;
    \item \textbf{compliance with point 2} : suppose point 2 is not valid, and
        then the encrypted message can be easily discovered by $ \Adv $. But
        then , as before, $ \Adv $ can win every game with $ \P [ 1 ]  $ , so
        the scheme couldn't be one time secure;
    \item \textbf{compliance with point 3} : suppose point 3 is not valid, and
        some information about $m$ is leaked in $c$, for example the first bit
        of $c$ is the same bit of $m$ . $\Adv $ could forge
        $m_{0}==m$ such that they have the same bits but just the first is
        different. When $\Adv $ obtains $c$, he can look at the first bit and
        distinguish which was the message encrypted. Thus, the scheme wouldn't
        be one time secure.
        TO BE REVIEWED.
\end{itemize}
What is not \textbf{two time secure} ?
\begin{construction}
    Let $G:\{0,1\}^{\lambda} \to \{0,1\}^{l} $. Consider the following schema
    $\Pi_{\xor}$:
    \begin{itemize}
        \item $\K=\{0,1\}^{\lambda} \Rightarrow k \leftarrow\$ \{0,1\}^{\lambda}$
        \item $Enc(k,m)=G(k) \xor m$, $m \in \{0,1\}^{l}$
        \item $Dec(k,c)=c \xor G(k)=m$
    \end{itemize}
\end{construction}
This construction isn't 2-time secure.Assume the pair
\[
    ( \bar{m}, \bar{c}=G(k) \xor \bar{m}) 
\]
is known. Now , given $c=G(k) \xor m$, where $c$ and $m$ are unknown, we can
force the schema and do the following 
\[
    \bar{c}=G(k) \xor m = c \xor m \xor \bar{m} \Rightarrow c \xor \bar{c} = m
    \xor \bar{m} 
\]
 and obtain $m$.

\begin{theorem}
    If $G$ is a PRG, then $\Pi_{\xor}$ is computationally one-time secure
\end{theorem}

\begin{proof}
    We need to show that 
    \[
        Game_{\Pi_{\xor}, \A}^{ind}(\lambda, 0)   \approx_{c} Game_{\Pi_{\xor},
        \A}^{ind}(\lambda, 1)
    \]

\begin{figure}[h!]
   \centering
   \sdinit{}
   \begin{tikzpicture}
      % Define symbols and names for the parties
      \sdbegin{}
      \newinst{A}{$ \A $}
      \newinst[5]{B}{$ C $} % Increase "5" to widen
      
      % Message from Alice to Bob, with precomputations
      \postlevel
      \mess{A}{$m_{0}, m \in \M$}{B}
      \node[anchor=west] at (mess to){};

      \postlevel
      \mess{B}{c}{A}
      \node[anchor=west] at (mess from) {\shortstack[l]{
                  $  k \leftarrow\$ \{0,1\}^{\lambda}  $ \\
            $ b \leftarrow\$ \{0,1\}$ \\
    $  c = G(k) \xor m_{b} $ }};
      
      % Message from Bob to Alice, with computations by both sides
      \postlevel
      \mess{A}{$b'$}{B}
      \node[anchor=west] at (mess to) {  };
      
      \sdend{}
   \end{tikzpicture}
   \caption{Game for $\Pi_{\xor}$ schema ($Game_{\Pi_{\xor}, \A}$)}
   \label{fig:orgame}
\end{figure}

So first consider the following \textbf{hybrid game} :
\begin{figure}[h!]
   \centering
   \sdinit{}
   \begin{tikzpicture}
      % Define symbols and names for the parties
      \sdbegin{}
      \newinst{A}{$ \A $}
      \newinst[5]{B}{$ C' $} % Increase "5" to widen
      
      % Message from Alice to Bob, with precomputations
      \postlevel
      \mess{A}{$m_{0}, m \in \M$}{B}
      \node[anchor=west] at (mess to){};

      \postlevel
      \mess{B}{c}{A}
      \node[anchor=west] at (mess from) {\shortstack[l]{
                  $  u \leftarrow\$ \{0,1\}^{\lambda}  $ \\
            $ b \leftarrow\$ \{0,1\}$ \\
    $  c = u \xor m_{b} $ }};
      
      % Message from Bob to Alice, with computations by both sides
      \postlevel
      \mess{A}{$b'$}{B}
      \node[anchor=west] at (mess to) {  };
      
      \sdend{}
   \end{tikzpicture}
   \caption{Hybrid game  ($\H\Y\B_{\Pi_{\xor}, \A}(\lambda, b)$)}
   \label{fig:hyb2}
\end{figure}

\hline
\begin{lemma}
    $\H\Y\B_{\Pi_{\xor}, \A}(\lambda, 0) \equiv \H\Y\B_{\Pi_{\xor},
    \A}(\lambda, 1)$
\end{lemma}
This is true because distribution of $c$ doesn't depend on $b \in \{0,1\}$.

\begin{lemma}
    $ \forall b \in \{0,1\}, \H\Y\B_{\Pi_{\xor}, \A}(\lambda, b) \approx_{c}
    Game_{\Pi, \A}^{ind}(\lambda, b)$
\end{lemma}

\begin{proof}
    Simple reduction to PRG, supposing that the statement isn't true.\\
    This means that there exists $\A'$ capable of distinguish $c=m_{b} \xor G(k)$
    and $c=m_{b} \xor u$.\\
    We will prove this first with $b=0$ (the other case is the same).
\newpage
\begin{figure}[h!]
   \centering
   \sdinit{}
   \begin{tikzpicture}
      % Define symbols and names for the parties
      \sdbegin{}
      \newinst{D}{$\A'$}
      \newinst[3]{A}{$ \A $}
      \newinst[3]{B}{$C^{prg} $} % Increase "5" to widen
      
      % Message from Alice to Bob, with precomputations
      \postlevel
      \mess{B}{$z$}{A}
      \node[anchor=west] at (mess from) {\shortstack[l]{
                  $  z \leftarrow\$ G(k) $ for $k \leftarrow\$ \{0,1\}^{\lambda}$
            \\
            $  z \leftarrow\$ U_{\lambda + l}  $ }};
      
      \mess{D}{$m_{0}, m$}{A}
      \node[anchor=west] at (mess to) {  };
      \postlevel
      \mess{A}{$c=m_{0} \xor z$}{D}
      \node[anchor=west] at (mess to) {  };
      % Message from Bob to Alice, with computations by both sides
      \postlevel
      \mess{D}{$(m_{0},z')$}{A}
      \node[anchor=west] at (mess to) {  };

       \mess{A}{$z'$}{B}
      \node[anchor=west] at (mess to) {  };
      
      \sdend{}
   \end{tikzpicture}
   \label{fig:asd1}
\end{figure}
If $\A'$ could distinguish these two sources, then $C^{prg}$ could be
distinguished, but this is impossible.
\end{proof}

Now , for the two lemmas just seen, we have
\[
    Game_{\Pi, \A}^{ind}(\lambda, 0) \approx_{c} 
    \H\Y\B_{\Pi_{\xor}, \A}(\lambda, 0) \equiv
    \H\Y\B_{\Pi_{\xor}, \A}(\lambda, 1) \approx_{c} 
     Game_{\Pi, \A}^{ind}(\lambda, 1)
\]

\end{proof}
 
\hline
\section{Pseudorandom functions}

A random function
\[  R:\{0,1\}^{n} \to \{0,1\}^{l} \].
is a function that takes in input $x$ and :
\begin{itemize}
    \item returns a new $R(x)=y \leftarrow\$ \{0,1\}^{l}$ if $x$ has never been
        saw before and records that value (so \textbf{ \textit{two distinct inputs can collide} })

    \item returns the recorded $R(x)$ otherwise
\end{itemize}.
We could generate these functions, but they occupy too much space: supposing all
the possible outputs of $R$ have been generated and stored in an array in memory, the
occupied bits in memory are $2^{n}l$.
\[  
    \color{black!50}%
    \overbracket[0.5pt][5pt]{{\thickspace}%
    \color{black}\printarray[4em]{0010...}\thickspace}^{l\textrm{ bits}}_{1}\thickspace%
    \overbracket[0.5pt][5pt]{{\thickspace}%
    \color{black}\printarray[4em]{1110...}\thickspace}^{l\textrm{ bits}}_{2}\thickspace%
    \overbracket[0.5pt][5pt]{{\thickspace}%
      \color{black}\printarray[4em]{0011...}\thickspace}^{l\textrm{ bits}}_{3}\thickspace%
    \color{black!25}\underbracket[0pt][5.5pt]{{\thickspace}%
      \printarray[2em]{...,...,...,...}\thickspace}_{}%
    \overbracket[0.5pt][5pt]{{\thickspace}%
      \color{black}\printarray[4em]{1111...}\thickspace}^{l\textrm{ bits}}_{2^{n}}\thickspace%
  \thickspace
\]
In particular, the family $\R=\{ R:\{0,1\}^{n} \to \{0,1\}^{l} \}$, also
indicated as $\R(\lambda, n, l)$,  containing
all the possible random functions has cardinality $2^{2^{n}l}$.\\

\textbf{Intuition:} a pseudo-random function is indistinguishable
(computationally speaking) from a truly random one.\\

Call $\F=\{ F_{k}:\{0,1\}^{n(\lambda)} \to \{0,1\}^{l(\lambda)} \}_{k \in
\{0,1\}^{\lambda}}$ the family of pseudorandom functions with key $k$. To give a
definition of pseudo-random function, consider the following games:




\begin{figure}[!h]
   \sdinit{}
   \begin{tikzpicture}[scale=0.5]
      % Define symbols and names for the parties
      \sdbegin{}
      \newinst{A}{$ \A $}
      \newinst[5]{B}{$ C $} % Increase "5" to widen
      \mess{A}{$x$}{B}
      \node[anchor=west] at (mess to) {  };
      \postlevel
      \mess{B}{z}{A}
      \node[anchor=west] at (mess from) {\shortstack[l]{
                  $ k \leftarrow\$ \{0,1\}^{\lambda}   $ 
            \\
    $ z=F_{k}(x)   $ }};
      % Message from Bob to Alice, with computations by both sides
      \postlevel
      \mess{A}{$b'$}{B}
      \node[anchor=west] at (mess to) {  };
      \sdend{}
   \end{tikzpicture}
   %\label{fig:}
   \sdinit{}
   \begin{tikzpicture}[scale=.5]
      % Define symbols and names for the parties
      \sdbegin{}
      \newinst{A}{$ \A $}
      \newinst[5]{B}{$ C $} % Increase "5" to widen
      \mess{A}{$x$}{B}
      \node[anchor=west] at (mess to) {  };
      \postlevel
      \mess{B}{z}{A}
      \node[anchor=west] at (mess from) {\shortstack[l]{
                  $ R \leftarrow\$ \R(\lambda, n, l) $ 
            \\
    $  z=R(x)  $ }};
      % Message from Bob to Alice, with computations by both sides
      \postlevel
      \mess{A}{$b'$}{B}
      \node[anchor=west] at (mess to) {  };
      \sdend{}
   \end{tikzpicture}
   \caption{$Real_{\F, \A}(\lambda)$ vs $Rand_{\R, \A}(\lambda) $}
   %\label{fig:}
\end{figure}

where $b' \in \{0,1\}$ is a convention and 1 is assigned to $Real$ or $Rand$; so
in this game the adversary \textbf{recognizes}  which machine he is talking with.

\begin{definition}
    $\F$ is a PRF family if 
    \[
        Real_{\F, \A}(\lambda) \approx_{c} Rand_{\R, \A}(\lambda)
    \]
\end{definition}

\begin{exercise}
    Show that no PRG is secure against \textbf{unbounded attackers} .
\end{exercise}

\begin{exercise}
    Show the same (as above) for PRF.
\end{exercise}

\subsection{GGM Tree}

\begin{construction}
    Let $G:\{0,1\}^{\lambda} \to \{0,1\}^{2\lambda} $ be a PRG and let us write 
    \[
        G(k)=(G_{0}(k), G_{1}(k))
    \]

    Now consider this tree, called \textbf{GGM tree}, which describes the use of
    $G(k)$:

    GGM TREE IMAGE


    Build $\F=\{ F_{k}:\{0,1\}^{n} \to \{0,1\}^{\lambda} \}$ such that 
    \[
        F_{k}(x)=G_{x_{n}}(G_{x_{n-1}}... G_{x_{2}}(G_{x_{1}}(k)))
    \]
\end{construction}

For example, in the tree with height $n=3$,  for $x=001$ we have $F_{k}(001)$, which is
$G_{0}(G_{0}(G_{1}(k)))$.

