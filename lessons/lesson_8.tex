\chapter*{Lesson 8}
\section{Domain extension}

How can we encrypt long messages , say $m=(m_{1}, m_{2}, ..., m_{t})$ where
for $i \in [t], m_{i} \in \{0,1\}^{n} $?\\

Mode of operation: let's build $P_{k}$, a blockcypher.\\

\subsection{Electronic CodeBook (ECB)}

(In notes, this name is associated with CTR-mode, or \textbf{counter mode}, but
why?)\\
The principle is that I have many blocks of information and I encrypt all of
them individually (maybe in parallel):
\[
    c_{i}=P_{k}(m_{i}) \forall i \in [t]
\]
and outputs $c_{1}, c_{2}, ..., c_{t}$.\\

If I encrypt the same message, I obtain the same cyphertext. So it's not
CPA-secure as the encryption function is \textbf{deterministic.}

\begin{figure}[h!]
\tikzstyle{vertex}=[draw,fill=black!15,circle,minimum size=20pt,inner sep=0pt];
\tikzstyle{encrypt}=[draw,fill=black!15,rectangle,minimum size=20pt,inner
sep=0pt];
\centering
\begin{tikzpicture}
    \newcommand{\n}{2}
    \foreach \nr in {0, ..., \n}{
        \node (r\nr) at (0,{(\n-\nr)*2}) {$r+\nr$};
        \node (c\nr) at (6,{(\n-\nr)*2}) {$c_{\nr+1}$};
        \node (F\nr)[encrypt] at (2,{(\n-\nr)*2}) {$F_{k}$};
        \node (m\nr) at (4,{(\n-\nr)*2+1}) {$m_{\nr}$};
        \node (x\nr) at (4,{(\n-\nr)*2}) {$\xor$};
        \draw[->,very thick] (r\nr) -- (F\nr);
        \draw[->,very thick] (m\nr) -- (x\nr);
        \draw[->,very thick] (F\nr) -- (x\nr);
        \draw[->,very thick] (x\nr) -- (c\nr);
    }
   \node (rx) at (0,-1) {$\vdots$};
   \node (Fx) at (2,-1) {$\vdots$};
   \node (cx) at (6,-1) {$\vdots$};
        \node (rt) at (0,{(-1)*2}) {$r + t - 1$};
        \node (ct) at (6,{(-1)*2}) {$c_{t-1}$};
        \node (Ft)[encrypt] at (2,{(-1)*2}) {$F_{k}$};
        \node (mt) at (4,{(-1)*2+1}) {$m_{t-1}$};
        \node (xt) at (4,{(-1)*2}) {$\xor$};
        \draw[->,very thick] (rt) -- (Ft);
        \draw[->,very thick] (mt) -- (xt);
        \draw[->,very thick] (Ft) -- (xt);
        \draw[->,very thick] (xt) -- (ct);
        \node (alala) at (6,6) {$c_{0}=r$};
        \draw[->,very thick] (r0) -- (alala);
\end{tikzpicture}
\end{figure}

In the image, each message is long exactly \textbf{n bits}, and 
\[
r \leftarrow\$ \{0,1\}^{n} 
\]
so the sums made over $r$ are done in \textbf{mod $2^{n}$}, since $r$ can be a
number in the range $[0, 2^{n} -1]$.\\

This is also called streamcipher, cause as input flows ECB produces the
ciphertext.How to decrypt this schema?\\
Since I know $r_{i}$, if I compute $F_{k}(r_{i})$ I can use the $\xor$-operation
to calculate the message.\\

\subsection{Cipher block chaining (CBC)}
================================================================================
================================================================================
                    CBC IMAGE
================================================================================
================================================================================
The schema to cipher here is 
\[ 
    \forall i \in [1,t] c_{i}=P_{k}(C_{i-1}\xor m_{1})
\]
How to decipher?
\[
    P_{k}^{-1}(i)= c_{i-1} \xor m_{i}
\]
so $m_{i}= P_{k}^{-1}(c_{i}) \xor c_{i-1}$.\\

\begin{theorem}
    Assume $\F$ ($F_{k}$ in the image above is part of this family) is a PRF,
    then CTR-mode yields a CPA-secure SKE for variable length messages.
\end{theorem}.\\
\textit{Variable length messages}  means that  every message 
\[
    m=(m_{1}, ..., m_{t})
\]

has t subsets, and $t$ can change from message $m$ to another message
$m'=(m'_{1}, ..., m'_{t'})$.
Consider the following hybrids:

\begin{figure}[h!]
   \centering
   \sdinit{}
   \begin{tikzpicture}[scale=0.65]
      % Define symbols and names for the parties
      \sdbegin{}
      \newinst{A}{$ \A $}
      \newinst[5]{B}{$ C^{cpa} $} % Increase "5" to widen
      
      \postlevel
      \mess{A}{$m=(m_{0}, ..., m_{t})$}{B}
      \node[anchor=west] at (mess to) {  };
      \postlevel
      \mess{B}{$c=(c_{0}=r, c_{1}, ..., c_{t})$}{A}
      \node[anchor=west] at (mess from) { \shortstack[l] {
                  $r \leftarrow\$ [r, r+t-1] mod 2^{n}$    
          \\
          $c_{i}=F_{k}(r + i -1) \xor m_{i}$
          }};
    \draw [->] (1.2,-3.3) to[out=240,in=110] (1.2,-1.7);
      
   \postlevel 
    \mess{A}{$m_{0}^{*}, m_{1}^{*}$}{B}
      \node[anchor=east] at (mess from) { both of length $t$ };

      \postlevel
      \mess{B}{$r^{*}, c_{1}^{*}, ..., c_{t}^{*}$}{A}
      \node[anchor=west] at (mess from) {\shortstack[l]{
        $r* \leftarrow\$ [r, r+t-1] mod 2^{n}$    
          \\
          $c_{i}^{*}=F_{k}(r^{*} + i -1) \xor m_{i}^{*}$
          }};
      
      % Message from Bob to Alice, with computations by both sides
      \postlevel
      \mess{A}{$m=(m_{0}, ..., m_{t})$}{B}
      \node[anchor=west] at (mess to) {  };
      \postlevel
      \mess{B}{$c$}{A}
      \node[anchor=west] at (mess to) {  };
    \draw [->] (1.2,-8.3) to[out=240,in=110] (1.2,-6.3);
      \postlevel
      \mess{A}{$b' \in \{0,1\}$}{B}
      \node[anchor=west] at (mess to) {  };
      \sdend{}
   \end{tikzpicture}
   \caption{$Game_{CTR, \A}^{cpa}(\lambda, b)$}
\end{figure}


\begin{figure}[h!]
   \centering
   \sdinit{}
   \begin{tikzpicture}[scale=0.5]
      % Define symbols and names for the parties
      \sdbegin{}
      \newinst{A}{$ \A $}
      \newinst[5]{B}{$ C^{cpa} $} % Increase "5" to widen
      
      \postlevel
      \mess{A}{$m=(m_{0}, ..., m_{t})$}{B}
      \node[anchor=west] at (mess to) {  };
      \postlevel
      \mess{B}{$c=(c_{0}=r, c_{1}, ..., c_{t})$}{A}
      \node[anchor=west] at (mess from) { \shortstack[l] {
                  $R \leftarrow\$ R(\lambda, n, n)$    
          \\
          $c_{i}=R(r + i -1) \xor m_{i}$
          }};
    \draw [->] (1.2,-3.3) to[out=240,in=110] (1.2,-1.7);
      
   \postlevel 
    \mess{A}{$m_{0}^{*}, m_{1}^{*}$}{B}
      \node[anchor=east] at (mess from) { both of length $t$ };

      \postlevel
      \mess{B}{$r^{*}, c_{1}^{*}, ..., c_{t}^{*}$}{A}
      \node[anchor=west] at (mess from) {\shortstack[l]{
        $r* \leftarrow\$ [r, r+t-1] mod 2^{n}$    
          \\
          $c_{i}^{*}=R(r^{*} + i -1) \xor m_{i}^{*}$
          }};
      
      % Message from Bob to Alice, with computations by both sides
      \postlevel
      \mess{A}{$m=(m_{0}, ..., m_{t})$}{B}
      \node[anchor=west] at (mess to) {  };
      \postlevel
      \mess{B}{$c$}{A}
      \node[anchor=west] at (mess to) {  };
    \draw [->] (1.2,-8.3) to[out=240,in=110] (1.2,-6.3);
      \postlevel
      \mess{A}{$b' \in \{0,1\}$}{B}
      \node[anchor=west] at (mess to) {  };
      \sdend{}
   \end{tikzpicture}
   \caption{$\H\Y\B_{1}(\lambda, b)$}
\end{figure}
\newpage
\begin{figure}[h!]
   \centering
   \sdinit{}
   \begin{tikzpicture}[scale=0.5]
      % Define symbols and names for the parties
      \sdbegin{}
      \newinst{A}{$ \A $}
      \newinst[5]{B}{$ C^{cpa} $} % Increase "5" to widen
      
      \postlevel
      \mess{A}{$m=(m_{0}, ..., m_{t})$}{B}
      \node[anchor=west] at (mess to) {  };
      \postlevel
      \mess{B}{$c=(c_{0}=r, c_{1}, ..., c_{t})$}{A}
      \node[anchor=west] at (mess from) { \shortstack[l] {
                  $c_{i} \leftarrow\$ U_{n}$    
          }};
    \draw [->] (1.2,-3.3) to[out=240,in=110] (1.2,-1.7);
      
   \postlevel 
    \mess{A}{$m_{0}^{*}, m_{1}^{*}$}{B}
      \node[anchor=east] at (mess from) { both of length $t$ };

      \postlevel
      \mess{B}{$c_{0}^{*}, c_{1}^{*}, ..., c_{t}^{*}$}{A}
      \node[anchor=west] at (mess from) {\shortstack[l]{
                  $c_{i}^{*} \leftarrow\$ U_{n}$    
          }};
      
      % Message from Bob to Alice, with computations by both sides
      \postlevel
      \mess{A}{$m=(m_{0}, ..., m_{t})$}{B}
      \node[anchor=west] at (mess to) {  };
      \postlevel
      \mess{B}{$c$}{A}
      \node[anchor=west] at (mess to) {  };
    \draw [->] (1.2,-8.3) to[out=240,in=110] (1.2,-6.3);
      \postlevel
      \mess{A}{$b' \in \{0,1\}$}{B}
      \node[anchor=west] at (mess to) {  };
      \sdend{}
   \end{tikzpicture}
   \caption{$\H\Y\B_{2}(\lambda, b)$}
\end{figure}

Now we want to show that $Game_{\Pi, \lambda}^{cpa}(\lambda,0)
\approx_{c}Game_{\Pi, \lambda}^{cpa}(\lambda,1) $
\begin{proof}

    \begin{exercise}
        \begin{lemma}
        Show that $Game_{\Pi, \lambda}^{cpa}(\lambda,b) \approx_{c} \H\Y\B_{1}
        (\lambda, b), \forall b \in \{0,1\} $
        \end{lemma}\\
        
        (Since this $Game(\lambda, b)$ is a CPA scheme and the second one is
        very similar, we can use a distinguisher which plays the CPA game; since
        this is a lemma, our precondition to "break" during the reduction is
        the precondition contained in the theorem statement)
    \end{exercise}

    \begin{lemma}
        $\H\Y\B_{1}(\lambda,b) \approx_{c} \H\Y\B_{2} (\lambda), \forall b \in
        \{0,1\}$
    \end{lemma}

    The two hybrids are identical but for the encryption function, which is
    completely random in the second one while the first one uses a random
    function.\\
    
    Since  $R(r + i ) \xor m_{i} \approx R(r + i ) $ (because
    $m_{i}$ doesn't affect the distribution of the result at all), if  $R(r^{*})$ 
     behaves like a true random extractor , the two hybrids are
    indistinguishable.\\

    Now, if I examine a simple query and the consecutive challenge query $(q,
    q^{*})$ in a game in $\H\Y\B_{2}$, the responses will be chosen at random
    with very low probability of being the same; but in a game in
    $\H\Y\B_{1}$, the problem comes if the challenger 
    creates an $r^{*}+ j'$ for $q^{*}$ which overlaps some chosen
    $r_{i} + j$ previously chosen for $q$.\\
    This is bad, because $R(r_{i} + j)=R(r^{*} + j')$ since $R$ is a random
    function, and the probability of outputting the same response ciphertext
    will be much higher in $\H\Y\B_{1}$.\\

    It's possible to show that these collisions happen very few times
    (with \textbf{negligible} probability) in $\H\Y\B_{1}$ and  that the two 
    hybrids are distinct and
    distinguishable for a negligible factor.\\

    \begin{proof}
        Let :
        \begin{itemize}
            \item $q$= $#$ of encryption queries
            \item $t_{i}$ = $#$ of blocks for the i-th query
            \item $t^{*}$ = $#$ of blocks for challenge
        \end{itemize}
 and let the function $R$ run: we will have 
 $R(r*), ...R(r^{*}+t^{*}-1)$ and $R(r_{i}), ..., R(r_{i}+t_{i}-1)$ sequences of random
 function outputs.

 \begin{definition}
     \textbf{OVERLAP event}: $\exists i, j, j', r_{i} + j=r^{*} + j'$ (here
     $j \leq t_{i}$ and $j' \leq t^{*}$), for some query $q_{i}$.
 \end{definition}
 
 If \textbf{OVERLAP}  does not happen , the sequence $(R(r^{*}), ...,
 R(r^{*}+t^{*}-1))$ is made of \textbf{ \textit{uniform} } and \textbf{
 \textit{independent}} values.
 Thus $\H\Y\B_{1}(b)$ is identical to $\H\Y\B_{2}(b)$ for all $b \in \{0,1\}$
 .\\

 Now it suffices to show that $ \P [ OVERLAP ] \in \negl{\lambda}  $.\\

 For simplicity, assume $q=$(length of each query) and also $t_{i}=t^{*}=q$.\\
Let $OVERLAP_{i}$ be the event that the $i-th$ query $r_{i}, ..., r_{i} + q - 1 $ 
partially or totally overlaps the challenge sequence $r^{*}, ..., r^{*} + q - 1 $.\\

Fix some $r*$.\\
One can see that $OVERLAP_{i}$ happens if 
\[
r^{*}-q+1 \leq r_{i} \leq r^{*} + q - 1
\]
, which means that $r_{i}$ should be chosen \textit{at least} in a way that :
\begin{itemize}
    \item the sequence $r^{*}, ..., r^{*} + q - 1 $ comes before the sequence $r_{i},
        ..., r_{i} + q - 1 $, and they overlap just for the last element  $r^{*}
        + q -1 = r_{i}$ or
    \item the sequence $r_{i},..., r_{i} + q - 1 $ comes before the sequence the
        sequence $r^{*}, ..., r^{*} + q - 1 $, and they overlap just for the
        last element $r_{i} + q - 1 = r^{*}$
\end{itemize}.\\
So now
\[
    \P [ OVERLAP_{i} ] = \frac{(r^{*} +q - 1) - (r^{*}-q+1) + 1}{2^{n}} =
    \frac{2q-1}{2^{n}}
\]

It is obvious that, for definition of \textbf{OVERLAP} , 
\[
    \P [ OVERLAP ] \leq \sum_{i=1}^{q} \P [ OVERLAP_{i} ] \leq 2
    \frac{q^{2}}{2^{n}} \in \negl{\lambda}  
\]


    \end{proof}
    
    Since $\H\Y\B_{1} \approx_{c} \H\Y\B_{2}$ and $\H\Y\B_{1}\equiv Game(b)$
    with $b$ equal to 0 and 1 respectively, we can state that 
    \[
        Game(0) \approx_{c} \H\Y\B_{2} \approx_{c} Game(1)
    \]
    .
    
\end{proof}

